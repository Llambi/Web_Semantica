\documentclass{llncs}   % llncs.cls esta en la ruta predefinida de TEX Live
\usepackage[colorlinks,citecolor=black,urlcolor=black,linkcolor=black,bookmarks=false,hypertexnames=true]{hyperref}
\begin{document}
\title{Información Semántica}

\author{Hugo Pérez Fernández.  \email{UO250708@uniovi.es}
\institute{Sistemas de Información para la Web. Grado de Ingeniería Informática. EII. 
universidad de Oviedo. Campus de los Catalanes. Oviedo}}
\maketitle              

\begin{abstract}

\keywords{Web \and Semántica \and JSON-LD\and microdatos \and RDF}
\end{abstract}

\section{Introducción}

\section{Asignación de entidades}
Se ha seleccionado el extracto (Anexo.\ref{noticia}) de la noticia sobre el ransomware que 
ha afectado a empresas de España 
(\href{https://www.elconfidencial.com/tecnologia/2019-11-04/everis-la-ser-ciberataque-ransomware_2312019/}{Noticia aquí}).
Y, usando la herramienta \href{https://dandelion.eu/semantic-text/entity-extraction-demo/}{Dandelion}, 
se han obtenido las siguientes entidades:

\begin{itemize}
    \item Organisations:
    \begin{itemize}
        \item Cadena SER.
        \item Prisa Radio.
        \item Everis.
        \item Radio Madrid.
    \end{itemize}
    \item Concepts:
    \begin{itemize}
        \item Ransomware.
        \item Virus Informatico.
        \item Internet.
        \item Computadora personal.
        \item Prensa escrita.
        \item Medio de Comunicación.
        \item Empresa.
    \end{itemize}
\end{itemize}

\section{JSON-LD}

\section{Anexos}
\subsection{Extracto de la noticia seleccionada.}\label{noticia}
Varias empresas españolas han sufrido hoy un serio ataque de 'ransomware' que recuerda al 
vivido a mediados de 2017 con Wannacry. Los primeros ataques confirmados de forma oficial 
los han sufrido la Cadena SER y otras emisoras de Prisa Radio, pero también varias consultoras 
tecnológicas, de las cuales Everis ha confirmado oficialmente estar afectada. "Estamos sufriendo 
un ataque masivo de virus a la red de Everis. Por favor, mantengan los PCs apagados". Es el 
mensaje interno que ha remitido Everis a sus empleados, según ha podido confirmar este diario 
y han publicado también varios medios especializados. La compañía confirma que ha enviado a sus 
trabajadores a casa hasta que puedan solventar la incidencia.

"PRISA Radio ha sufrido esta madrugada un ataque de virus que ha tenido una afectación grave y 
generalizada de todos nuestros sistemas informáticos. Los técnicos especializados en este tipo 
de situaciones aconsejan encarecidamente la desconexión total de todos los sistemas con el fin 
de evitar la propagación del virus. Hablamos, por tanto, de una situación de extrema emergencia", 
ha comunicado esta mañana la empresa en un mensaje interno al que ha tenido acceso este diario.

"A partir de ese momento, se irá chequeando puesto a puesto -siguiendo las instrucciones precisas 
que os enviará el departamento de Sistemas- para autorizar en los casos en que haya garantía 
absoluta la puesta en marcha de cada equipo. Por el momento, y hasta nuevo aviso, la emisión de 
Cadena SER queda centralizada en Radio Madrid; quedan anuladas todas las emisiones locales y 
regionales, salvo aquellas que hayan sido autorizadas expresamente por la Dirección de Antena. 
En este momento, la seguridad es la máxima de la compañía por encima de cualquier otro compromiso. 
Cualquier acción individual no autorizada puede poner en peligro el trabajo de rescate que se está 
llevando a cabo", cierra el mensaje.
\end{document}