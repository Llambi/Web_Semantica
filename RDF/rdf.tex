\documentclass[spanish]{llncs}   % llncs.cls esta en la ruta predefinida de TEX Live
\usepackage[colorlinks,citecolor=black,urlcolor=black,linkcolor=black,bookmarks=false,hypertexnames=true]{hyperref}
\usepackage{listingsutf8}
\usepackage{graphicx}


\begin{document}
\title{Práctica 9 - Tripletas RDF}

\author{Hugo Pérez Fernández.  \email{UO250708@uniovi.es}
\institute{Sistemas de Información para la Web. Grado de Ingeniería Informática. EII. 
universidad de Oviedo. Campus de los Catalanes. Oviedo.}}
\maketitle              

\section{Introducción}


\section{Obtención manual de información estructurada}

Primero obtendremos toda la información estructurada de los textos indicados en el Anexo.\ref{Textos}, usando para ello la web \href{https://schema.org}{Schema.org}:

\paragraph{Texto1}

En el caso de este texto se esta mecionando a una persona (\textit{schema:Person}), cuyo nombre es Miles Davis, ademas esta persona tiene como nacionalidad \textit{schema:Nacionality}
Americano (Estadounidense), y su profesión (\textit{schema:Ocupation}) es músico concretamente de Jazz.

\paragraph{Texto2}

En en este caso la información se puede estructurar de la siguiente forma:

\begin{itemize}
    \item Hay y una persona (\textit{schema:Person}) que se llama Barack Obama cuya ocupación (\textit{schema:Ocupation}) es presidente.
    \item Hay una organizacion (\textit{schema:Organization}) que se llama Unión Europea.
    \item Hay una ciudad (\textit{schema:City}) que se llama Washington.
    \item Hay una tipo de cambio de moneda (\textit{schema:ExchangeRateSpecification}) que se llama Euro y tiene un valor de 1.3.
    \item Hay una reunion (\textit{schema:BusinessEvent}) entre Barack Obama y la Unión Europea
    sobre el el valor del tipo de cambio del Euro.
\end{itemize}

\paragraph{Texto3}

En este último se  puede estructura la información del siguiente modo:

\begin{itemize}
    \item Hay una organización (\textit{schema:Organization}) que se llama The New York Times.
    \item Hay una persona (\textit{schema:Person}) que se llama Jhon McCarthy.
    \item Jhon McCarthy está muerto.
    \item El New York Times informa (\textit{schema:Report}) que Jhon McCarthy esta muerto.
    \item LISP es un lenguaje de programación.
    \item John McCarthy inventó LISP.
\end{itemize}

Con los datos obtenidos anteriormente se, usando la notacion Turtle, la información estructurada queda de la siguiente manera:

\lstset{
    frame=tb, % draw a frame at the top and bottom of the code block
    tabsize=2, % tab space width
    showstringspaces=false, % don't mark spaces in strings
    numbers=left, % display line numbers on the left
    commentstyle=\color{green}, % comment color
    keywordstyle=\color{black}, % keyword color
    stringstyle=\color{blue}, % string color    
    inputencoding=utf8/latin1
}
\lstinputlisting[language=HTML, breaklines, caption={Archivo Turtle con la información estructurada de los textos.}]{resources/texts.ttl}

\section{Obtención automática de información estructurada}

\section{Anexos}

\subsection{Textos de Prueba}\label{Textos}

“Miles Davis was an american jazz musician.”

“President Barack Obama and European Union leaders huddled in Washington amid growing fears over the future of the euro, which closed greater than 1.3 dollars.”

“The New York Times reported that John McCarthy died. He invented the programming language LISP.”

\end{document}